\documentclass{thesis}
\bibliographystyle{unsrt}
%%% jdummy.def
%
\DeclareRelationFont{JY1}{mc}{it}{}{OT1}{cmr}{it}{}
\DeclareRelationFont{JT1}{mc}{it}{}{OT1}{cmr}{it}{}
\DeclareFontShape{JY1}{mc}{m}{it}{<5> <6> <7> <8> <9> <10> sgen*min
    <10.95><12><14.4><17.28><20.74><24.88> min10
    <-> min10}{}
\DeclareFontShape{JT1}{mc}{m}{it}{<5> <6> <7> <8> <9> <10> sgen*tmin
    <10.95><12><14.4><17.28><20.74><24.88> tmin10
    <-> tmin10}{}
\DeclareRelationFont{JY1}{mc}{sl}{}{OT1}{cmr}{sl}{}
\DeclareRelationFont{JT1}{mc}{sl}{}{OT1}{cmr}{sl}{}
\DeclareFontShape{JY1}{mc}{m}{sl}{<5> <6> <7> <8> <9> <10> sgen*min
    <10.95><12><14.4><17.28><20.74><24.88> min10
    <-> min10}{}
\DeclareFontShape{JT1}{mc}{m}{sl}{<5> <6> <7> <8> <9> <10> sgen*tmin
    <10.95><12><14.4><17.28><20.74><24.88> tmin10
    <-> tmin10}{}
\DeclareRelationFont{JY1}{mc}{sc}{}{OT1}{cmr}{sc}{}
\DeclareRelationFont{JT1}{mc}{sc}{}{OT1}{cmr}{sc}{}
\DeclareFontShape{JY1}{mc}{m}{sc}{<5> <6> <7> <8> <9> <10> sgen*min
    <10.95><12><14.4><17.28><20.74><24.88> min10
    <-> min10}{}
\DeclareFontShape{JT1}{mc}{m}{sc}{<5> <6> <7> <8> <9> <10> sgen*tmin
    <10.95><12><14.4><17.28><20.74><24.88> tmin10
    <-> tmin10}{}
\DeclareRelationFont{JY1}{gt}{it}{}{OT1}{cmbx}{it}{}
\DeclareRelationFont{JT1}{gt}{it}{}{OT1}{cmbx}{it}{}
\DeclareFontShape{JY1}{mc}{bx}{it}{<5> <6> <7> <8> <9> <10> sgen*goth
    <10.95><12><14.4><17.28><20.74><24.88> goth10
    <-> goth10}{}
\DeclareFontShape{JT1}{mc}{bx}{it}{<5> <6> <7> <8> <9> <10> sgen*tgoth
    <10.95><12><14.4><17.28><20.74><24.88> tgoth10
    <-> tgoth10}{}
\DeclareRelationFont{JY1}{gt}{sl}{}{OT1}{cmbx}{sl}{}
\DeclareRelationFont{JT1}{gt}{sl}{}{OT1}{cmbx}{sl}{}
\DeclareFontShape{JY1}{mc}{bx}{sl}{<5> <6> <7> <8> <9> <10> sgen*goth
    <10.95><12><14.4><17.28><20.74><24.88> goth10
    <-> goth10}{}
\DeclareFontShape{JT1}{mc}{bx}{sl}{<5> <6> <7> <8> <9> <10> sgen*tgoth
    <10.95><12><14.4><17.28><20.74><24.88> tgoth10
    <-> tgoth10}{}
\DeclareRelationFont{JY1}{gt}{sc}{}{OT1}{cmbx}{sc}{}
\DeclareRelationFont{JT1}{gt}{sc}{}{OT1}{cmbx}{sc}{}
\DeclareFontShape{JY1}{mc}{bx}{sc}{<5> <6> <7> <8> <9> <10> sgen*goth
    <10.95><12><14.4><17.28><20.74><24.88> goth10
    <-> goth10}{}
\DeclareFontShape{JT1}{mc}{bx}{sc}{<5> <6> <7> <8> <9> <10> sgen*tgoth
    <10.95><12><14.4><17.28><20.74><24.88> tgoth10
    <-> tgoth10}{}
\DeclareRelationFont{JY1}{gt}{it}{}{OT1}{cmr}{it}{}
\DeclareRelationFont{JT1}{gt}{it}{}{OT1}{cmr}{it}{}
\DeclareFontShape{JY1}{gt}{m}{it}{<5> <6> <7> <8> <9> <10> sgen*goth
    <10.95><12><14.4><17.28><20.74><24.88> goth10
    <-> goth10}{}
\DeclareFontShape{JT1}{gt}{m}{it}{<5> <6> <7> <8> <9> <10> sgen*tgoth
    <10.95><12><14.4><17.28><20.74><24.88> tgoth10
    <-> tgoth10}{}
\endinput
%%%% end of jdummy.def

 % フォント関連のエラー対策(らしい)
\usepackage[dvipdfmx]{graphicx}
\usepackage{amsmath}			% math系
\usepackage{amssymb}			% math系
%\usepackage{float}				% 図表の挿入箇所を固定する[H]指定
\usepackage{cite}				% 参考文献
%\usepackage{url}				% 参考文献中のURL表記
\usepackage{algorithm}			% アルゴリズム環境
\usepackage{algorithmic}		% アルゴリズム環境
\usepackage{comment}			% コメントアウト環境
\usepackage{bm}	%太字形式のベクトル
\usepackage{amsthm}%定理用?
%%% 泉先生がコメントをつける用 %%%
\usepackage[normalem]{ulem}
\usepackage{color}
\newcommand{\Izumi}[1]{\textcolor{blue}{#1}}
\newcommand{\Izurep}[2]{\textcolor{red}{\sout{#1}}{\Izumi{#2}}}

\headsep=1.4cm  %本文上にスペースを空けたい場合は 20mm にする

% 定理環境
\usepackage{amsthm} %定理用
\theoremstyle{definition}
\newtheorem{theorem}{定理}[chapter]
\newtheorem{lemma}{補題}[chapter]
\newtheorem{definition}{定義}[chapter]
\newtheorem{fact}{事実}[chapter]
\newtheorem*{prf*}{証明}
%\renewcommand{\theproof}{}
%\newcommand{\qed}{\hfill$\square$\par}

%%%%%%%% ここから本体 %%%%%%%%%%%%%%%%%%%%%%%%

\begin{document}
\baselineskip=22pt
\pagestyle{empty}

% タイトル
\gradyear{30}
\papertitle{最大$k$-plex問題における \\ 準指数時間アルゴリズム}
\IDNumber{27115067}
\department{情報工学科}
\labo{泉研究室}
\enteryear{27}
\name{佐藤 僚祐}
\maketitle

% 目次
\pagestyle{myheadings}	% ページ番号を右上につける
\pagenumbering{roman}	% ページ番号をローマ数字で
\tableofcontents

\newpage

% 本文
\pagenumbering{arabic}	% ページ番号をアラビア数字で

\chapter{はじめに}

\section{研究背景}
近年,凝集性のグラフの検出はソーシャルネットワークの分野において
大きな注目を集めている.凝集性のグラフのひとつにクリークがある.
しかし,クリークは制約が厳しく扱いづらい面がある.そこで,代わりに
$k$-plexが用いられる場合があり,$k$-plexはクリークの緩和モデルと
なっている.グラフ中の最も大きい$k$-plexを検出する問題を
最大$k$-plex問題という.最大$k$-plex問題はNP完全であり,
頂点数$n$に対して$n$の多項式時間で解くことができないことが知られている.

現在,最大$k$-plex問題を$\sigma_{k}^{n}n^{O(1) }$ ($\sigma_{k} < 2$は$k$に
関する値)で解くアルゴリズムがあることが知られている.これは$2^{n}$という
理論的限界を破った最初のアルゴリズムである.この論文では最大$k$-plex問題を
,辺の本数$m$に対して$n^{k }2^{O(\sqrt{m})}$アルゴリズムを提案する.この
アルゴリズムも同様に$2^{n}$という理論的限界を破るアルゴリズムである.

\section{本研究の成果}
本研究では,最大2-plex問題を$n^{2}2^{O(\sqrt{m}})$で解くアルゴリズムを
提案する.さらにその発展として,最大$k$-plex問題を$n^{k }2^{O(\sqrt{m})}$で
解くアルゴリズムを提案する.

\section{論文の構成}
本論文は全4章で構成される.第2章ではグラフの構造と用語の定義をしている.
第3章では既存の最大クリーク問題におけるアルゴリズムの導入とそれを
応用した最大2-plex問題と最大$k$-plex問題のアルゴリズムの提案を行っている.
第4章ではまとめと今後の課題について述べている.

\newpage

\chapter{諸定義}

\section{グラフの構造}
本論文中のグラフ$G=(V,E)$は頂点数$n=|V|$と辺の本数$m=|E|$を持つ単純無向
グラフとする.グラフ$G$の頂点部分集合$S$によって誘導される誘導部分グラフを
$G[S]$と表す.頂点$v$に辺が接続されている頂点を$v$の近傍と呼び,$N(v)$と表す.
頂点$(u,v)$間の最短パスの本数を頂点$(u,v)$の距離とする.
%頂点$v$から距離$l$にある頂点を$v_l$と表し,その数を$|v_l|$とする.

\section{最大クリークと最大$k$-plex}
グラフ$G$の頂点部分集合$S$が完全グラフとき,$S$をクリークといい,
クリークのうち最大サイズのものを見つける問題を最大クリーク問題という.
一方,$G[S]$の全ての頂点が少なくとも$|S| - k$の次数を持つとき,
$S$を$k$-plexといい,$k$-plexのうち最大サイズのものを見つける問題を
最大$k$-plex問題という.

%\section{準指数時間アルゴリズム}
%$o(n)$アルゴリズム

\newpage

\chapter{最大2-plex問題と最大$k$-plex問題の \\ 準指数時間アルゴリズム}

\section{最大クリーク問題}
%コメントここから
\begin{comment}
以下に,既存の$2^{O(\sqrt{m})}$アルゴリズムを示す.
最小次数の頂点$v$を選ぶ.もし$v$の次数が$\sqrt{2m}$以上であるならば
$n \leqq \sqrt{2m}$であるから総当たり法をすると$n2^{O(n)} = n2^{O(\sqrt{m})}$である.
以降の$v$の次数は$\sqrt{2m}$以下であるとする.
$G$中の最大サイズのクリーク$C$を見つける単純なブランチングアルゴリズムを考える.
最小次数の頂点$v$を選んで,(i)$v$が$C$の頂点の一つである場合と
(ii)$v$が$C$の頂点の一つでない場合の二つの部分問題にブランチする.
(i)では,ブルートフォース法を用いて$v$の近傍にある最大サイズのクリーク$C'$を見つける.
$N(v) \leqq \sqrt{2m}$であるから$C'$に入る可能性のある部分集合の個数は多くとも
$2^{\sqrt{2m}}$であり、これが(i)におけるアルゴリズムのステップ数である.
(ii)では,$G$から$v$を削除して再帰的にアルゴリズムを呼び出す.最大$n$回呼び出す可能性がある.
(i)(ii)よりアルゴリズムの合計のステップ数は $n2^{\sqrt{2m}}$であり,アルゴリズムの実行時間は
 $2^{O(\sqrt{m})}$である.
\end{comment}
%コメントここまで
以下に,既存の最大クリーク問題に関する定理とその証明を示す.
\begin{theorem} \label{theorem:1}
$m$本の辺を持ったグラフの最大クリーク問題は$2^{O(\sqrt{m})}$時間で解くことができる.
\begin{prf*}
$G$を連結グラフとする.最小次数の頂点$v$を選ぶ.もし$v$の次数が$\sqrt{2m}$以上であるならば,
\[ 2m = \sum_{v \in V}^{} |N(v)| \geqq n\sqrt{2m} \]
であるから$n \leqq \sqrt{2m}$である.全ての可能性のある頂点集合を計算する
ブルートフォースアルゴリズムの実行時間は$n2^{O(n)} = n2^{O(\sqrt{m})}$となる.
以下$v$の次数は$\sqrt{2m}$未満であるとする.

$G$中の最大サイズのクリーク$C$を見つけるブランチングアルゴリズムを考える.
以下のような二つの部分問題にブランチする.
\begin{enumerate}
 \item $v$が$C$中の頂点の一つである
 \item $v$が$C$中の頂点の一つでない
\end{enumerate}
一つ目の部分問題では$|N(v)|$の全ての頂点部分集合をブルートフォースによって
探索して$v$を含んだ最大サイズのクリークを見つける.二つ目の部分問題では$v$を
$G$から削除してアルゴリズムを再帰的に呼び出す.

このアルゴリズムのステップ数が多くとも$n2^{\sqrt{2m}}$になることを$m$への帰納法によって証明する.
\begin{itemize}
 \item $m = 0$のとき,$G$はただ一つのクリークを持ち,その大きさは1である.	
 \item $v \in C$の場合に対応した部分問題を解くために,ブルートフォースを使って
$|N(v)|$中の最大サイズのクリーク$C'$を選ぶ.$C = C'  \cup  v$である.$|N(v)| < \sqrt{2m}$で
あるから,必要なステップ数は多くとも$2^{\sqrt{2m}}$ステップである.
 \item $v \notin C$の場合,帰納法の仮定によって問題は多くとも$(n - 1)2^{\sqrt{2m}}$ステップで解ける.
\end{itemize}
ゆえにアルゴリズムの合計のステップ数は
\[  2^{\sqrt{2m}} +  (n - 1)2^{\sqrt{2m}} = n2^{\sqrt{2m}} \]
でありその実行時間は $2^{O(\sqrt{m})}$である.
\end{prf*}
\end{theorem}

\section{最大2-plex問題}
最大2-plex問題の準指数時間アルゴリズムを提案するために,以下に2つの補題を示す.
\begin{lemma} \label{lemma:1}
$S$を2-plexの頂点集合とする.ある頂点$v$が$S$に含まれているとき,
$v$から距離が3以上離れた頂点は$S$に含まれない.
\begin{prf*}
$v$は2-plexの頂点集合$S$に含まれているとする.
$v$から距離が3離れた頂点を$a$,$v$から距離が2離れており$a$と隣接している
頂点を$b$とする.2-plexの定義より,$S$中に含まれる全ての頂点は少なくとも
$|S| - 2$の次数を持つ,つまり$S$中に含まれる頂点は$S$中の$|S| - 1$頂点と
隣接しているはずである.もし$a$が$S$中に含まれる頂点であると仮定すると,
$v$は$a$とも$b$とも隣接していないため矛盾する.したがって$a$は$S$に
含まれない.距離が4以上離れている頂点についても同様である.
\end{prf*}
\end{lemma}
%定理環境内で英数字が斜体になる点を後で直す
\begin{lemma} \label{lemma:2}
$S$を2-plexの頂点集合とする.ある頂点$v$が$S$に含まれているとき,
$v$から距離が2離れた頂点は二つ以上は$S$に含まれない.
\begin{prf*}
$v$は2-plexの頂点集合$S$に含まれているとする.
$v$から距離が2離れた二つの頂点をそれぞれ$a$,$b$とする.$S$中に含まれる頂点は
$S$中の$|S| - 1$頂点と隣接しているはずである.もし$a$と$b$が両方とも$S$中に
含まれると仮定すると,$v$は$a$とも$b$とも隣接していないため矛盾する.したがって
$a$と$b$は両方とも$S$に含まれることはない.
\end{prf*}
\end{lemma}
これらの補題をふまえ,最大2-plex問題に関する定理とその証明を示す.
\begin{theorem} \label{theorem:2}
$m$本の辺を持ったグラフの最大2-plex問題は$2^{O(\sqrt{m})}$時間で解くことができる.
\begin{prf*}
$G$を連結グラフとする.最小次数の頂点$v$を選ぶ.定理 \ref{theorem:1} と同様の議論により
$v$の次数は$\sqrt{2m}$未満であるとする.

$G$中の最大サイズの2-plexである$S$を見つけるために二つの部分問題にブランチする.
\begin{enumerate}
 \item $v$が$S$中の頂点の一つである
 \item $v$が$S$中の頂点の一つでない
\end{enumerate}
一つ目の部分問題では$v$から距離2以内にあり,$S$に含まれる可能性のある
頂点部分集合を探索して$v$を含んだ最大2-plexを見つける.
二つ目の部分問題では$v$を$G$から削除してアルゴリズムを再帰的に呼び出す.

このアルゴリズムのステップ数が多くとも$n^{2}2^{\sqrt{2m}}$になることを$m$への帰納法によって証明する.
\begin{itemize}
 \item $m = 0$のとき,$G$はただ一つの2-plexを持ち,その大きさは1である.	
 \item $v \in C$の場合に対応した部分問題を解くために,$v$を含んだ最大サイズの2-plexである$S$を選ぶ.
補題  \ref{lemma:1} より,$v$から距離3以上離れている頂点は$S$に含まれない.
補題  \ref{lemma:2} より,$v$から距離2離れている頂点は二つ以上は$S$に含まれない.
$v$の近傍の頂点数を$M$,$v$から距離2離れた頂点数を$N$とすると,$S$を選ぶのに必要な頂点数は
$2^{M} \times N$である.$M < \sqrt{2m}$,$N \leqq n$であるから,必要なステップ数は
多くとも$2^{\sqrt{2m}}$ステップである.
 \item $v \notin C$の場合,帰納法の仮定によって問題は多くとも$(n - 1)n2^{\sqrt{2m}}$ステップで解ける.
\end{itemize}
ゆえにアルゴリズムの合計のステップ数は
\[  n2^{\sqrt{2m}} +  (n - 1)n2^{\sqrt{2m}} = n^{2}2^{\sqrt{2m}} \]
でありその実行時間は $2^{O(\sqrt{m})}$である.
\end{prf*}
\end{theorem}

\section{最大$k$-plex問題}
結果どうこう

\newpage

\chapter{まとめと今後の課題}
\section{まとめ}
まとめました
\section{今後の課題}
こうこうこうがかだいです

\chapter*{謝辞}
本研究の機会を与え,数々の御指導を賜りました泉泰介准教授に深く感謝致します.
また,本研究を進めるにあたり多くの助言を頂き,様々な御協力を頂きました泉研究室
の学生のみなさんに深く感謝致します.

\chapter*{参考文献}
あんまりない

\end{document}