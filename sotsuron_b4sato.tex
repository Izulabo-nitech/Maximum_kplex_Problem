\documentclass{thesis}
\bibliographystyle{unsrt}
%%% jdummy.def
%
\DeclareRelationFont{JY1}{mc}{it}{}{OT1}{cmr}{it}{}
\DeclareRelationFont{JT1}{mc}{it}{}{OT1}{cmr}{it}{}
\DeclareFontShape{JY1}{mc}{m}{it}{<5> <6> <7> <8> <9> <10> sgen*min
    <10.95><12><14.4><17.28><20.74><24.88> min10
    <-> min10}{}
\DeclareFontShape{JT1}{mc}{m}{it}{<5> <6> <7> <8> <9> <10> sgen*tmin
    <10.95><12><14.4><17.28><20.74><24.88> tmin10
    <-> tmin10}{}
\DeclareRelationFont{JY1}{mc}{sl}{}{OT1}{cmr}{sl}{}
\DeclareRelationFont{JT1}{mc}{sl}{}{OT1}{cmr}{sl}{}
\DeclareFontShape{JY1}{mc}{m}{sl}{<5> <6> <7> <8> <9> <10> sgen*min
    <10.95><12><14.4><17.28><20.74><24.88> min10
    <-> min10}{}
\DeclareFontShape{JT1}{mc}{m}{sl}{<5> <6> <7> <8> <9> <10> sgen*tmin
    <10.95><12><14.4><17.28><20.74><24.88> tmin10
    <-> tmin10}{}
\DeclareRelationFont{JY1}{mc}{sc}{}{OT1}{cmr}{sc}{}
\DeclareRelationFont{JT1}{mc}{sc}{}{OT1}{cmr}{sc}{}
\DeclareFontShape{JY1}{mc}{m}{sc}{<5> <6> <7> <8> <9> <10> sgen*min
    <10.95><12><14.4><17.28><20.74><24.88> min10
    <-> min10}{}
\DeclareFontShape{JT1}{mc}{m}{sc}{<5> <6> <7> <8> <9> <10> sgen*tmin
    <10.95><12><14.4><17.28><20.74><24.88> tmin10
    <-> tmin10}{}
\DeclareRelationFont{JY1}{gt}{it}{}{OT1}{cmbx}{it}{}
\DeclareRelationFont{JT1}{gt}{it}{}{OT1}{cmbx}{it}{}
\DeclareFontShape{JY1}{mc}{bx}{it}{<5> <6> <7> <8> <9> <10> sgen*goth
    <10.95><12><14.4><17.28><20.74><24.88> goth10
    <-> goth10}{}
\DeclareFontShape{JT1}{mc}{bx}{it}{<5> <6> <7> <8> <9> <10> sgen*tgoth
    <10.95><12><14.4><17.28><20.74><24.88> tgoth10
    <-> tgoth10}{}
\DeclareRelationFont{JY1}{gt}{sl}{}{OT1}{cmbx}{sl}{}
\DeclareRelationFont{JT1}{gt}{sl}{}{OT1}{cmbx}{sl}{}
\DeclareFontShape{JY1}{mc}{bx}{sl}{<5> <6> <7> <8> <9> <10> sgen*goth
    <10.95><12><14.4><17.28><20.74><24.88> goth10
    <-> goth10}{}
\DeclareFontShape{JT1}{mc}{bx}{sl}{<5> <6> <7> <8> <9> <10> sgen*tgoth
    <10.95><12><14.4><17.28><20.74><24.88> tgoth10
    <-> tgoth10}{}
\DeclareRelationFont{JY1}{gt}{sc}{}{OT1}{cmbx}{sc}{}
\DeclareRelationFont{JT1}{gt}{sc}{}{OT1}{cmbx}{sc}{}
\DeclareFontShape{JY1}{mc}{bx}{sc}{<5> <6> <7> <8> <9> <10> sgen*goth
    <10.95><12><14.4><17.28><20.74><24.88> goth10
    <-> goth10}{}
\DeclareFontShape{JT1}{mc}{bx}{sc}{<5> <6> <7> <8> <9> <10> sgen*tgoth
    <10.95><12><14.4><17.28><20.74><24.88> tgoth10
    <-> tgoth10}{}
\DeclareRelationFont{JY1}{gt}{it}{}{OT1}{cmr}{it}{}
\DeclareRelationFont{JT1}{gt}{it}{}{OT1}{cmr}{it}{}
\DeclareFontShape{JY1}{gt}{m}{it}{<5> <6> <7> <8> <9> <10> sgen*goth
    <10.95><12><14.4><17.28><20.74><24.88> goth10
    <-> goth10}{}
\DeclareFontShape{JT1}{gt}{m}{it}{<5> <6> <7> <8> <9> <10> sgen*tgoth
    <10.95><12><14.4><17.28><20.74><24.88> tgoth10
    <-> tgoth10}{}
\endinput
%%%% end of jdummy.def

 				% フォント関連のエラー対策(らしい)
\usepackage[dvipdfmx]{graphicx}
\usepackage{amsmath}			% math系
\usepackage{amssymb}			% math系
%\usepackage{float}				% 図表の挿入箇所を固定する[H]指定
\usepackage{cite}				% 参考文献
%\usepackage{url}				% 参考文献中のURL表記
\usepackage{algorithm}			% アルゴリズム環境
\usepackage{algorithmic}			% アルゴリズム環境
\usepackage{comment}			% コメントアウト環境
\usepackage{bm}					%太字形式のベクトル
\usepackage{amsthm}			%定理用?
%%% 泉先生がコメントをつける用 %%%
\usepackage[normalem]{ulem}
\usepackage{color}
\newcommand{\Izumi}[1]{\textcolor{blue}{#1}}
\newcommand{\Izurep}[2]{\textcolor{red}{\sout{#1}}{\Izumi{#2}}}

\headsep=1.4cm  %本文上にスペースを空けたい場合は 20mm にする

% 定理環境
\usepackage{amsthm} %定理用
\theoremstyle{definition}
\newtheorem{theorem}{定理}[chapter]
\newtheorem{lemma}{補題}[chapter]
\newtheorem{definition}{定義}[chapter]
\newtheorem{fact}{事実}[chapter]
\newtheorem*{prf*}{証明}
%\renewcommand{\theproof}{}
%\newcommand{\qed}{\hfill$\square$\par}

%%%%%%%% ここから本体 %%%%%%%%%%%%%%%%%%%%%%%%

\begin{document}
\baselineskip=22pt
\pagestyle{empty}

% タイトル
\gradyear{30}
\papertitle{最大$k$-plex問題における \\ 準指数時間アルゴリズム}
\IDNumber{27115067}
\department{情報工学科}
\labo{泉研究室}
\enteryear{27}
\name{佐藤 僚祐}
\maketitle

% 目次
\pagestyle{myheadings}	% ページ番号を右上につける
\pagenumbering{roman}	% ページ番号をローマ数字で
\tableofcontents

\newpage

% 本文
\pagenumbering{arabic}	% ページ番号をアラビア数字で

\chapter{はじめに}

\section{研究背景}
近年,凝集性のグラフの検出はソーシャルネットワークの分野において
大きな注目を集めている.凝集性のグラフのひとつにクリークがある.
しかし,クリークは制約が厳しく扱いづらい面がある.そこで,代わりに
$k$-plexが用いられる場合があり,$k$-plexはクリークの緩和モデルと
なっている.グラフ中の最も大きい$k$-plexを検出する問題を
最大$k$-plex問題という.最大$k$-plex問題はNP完全であり,
頂点数$n$に対して$n$の多項式時間で解くことができないことが知られている.
この論文では最大$k$-plex問題を,辺の本数$m$に対して$n^{k }2^{O(\sqrt{m})}$アルゴリズムを提案する.

\section{関連研究}
最大$k$-plex問題を解くためのアルゴリズムとして,以下のようなものが研究されている.
\begin{itemize}
 \item 最大$k$-plex問題を整数計画法で定式化してbranch-and-cutを
	用いるアルゴリズムの研究 \cite{balasundaram2011clique}
 \item グラフ彩色問題の概念を一般化して$k$-plexの基数の上限を導いて
	それを利用した組み合わせアルゴリズムの研究 \cite{mcclosky2012combinatorial}
 \item 最大$k$-plex問題と双対性を持つ$d$-BDD(Bounded-Degree-$d$ Vertex Detection)問題の
	解を見つけるアルゴリズムの研究 \cite{moser2012exact}
\end{itemize}
上記の研究で提案されている最大$k$-plex問題を解くためのアルゴリズムは
全て$2^{n}n^{O(1)}$時間で動く.

また,複数のブランチングルールを持ったbranch-and-searchアルゴリズムによって
最大$k$-plex問題を$\sigma_{k}^{n}n^{O(1) }$ ($\sigma_{k} < 2$は$k$に関する値)で解く
アルゴリズムがある. \cite{xiao2017fast}  これは$2^{n}$という理論的限界を破った最初のアルゴリズムである.

\section{本研究の成果}
本研究では,,既存の$2^{O(\sqrt{m})}$時間の最大クリーク問題のアルゴリズムを
応用して最大2-plex問題を$n^{2}2^{O(\sqrt{m}})$で解くアルゴリズムを提案する.
さらにそのアルゴリズムを拡張させて,最大$k$-plex問題を$n^{k }2^{O(\sqrt{m})}$で
解くアルゴリズムを提案する.

\section{論文の構成}
本論文は全4章で構成される.第2章ではグラフの構造と用語の定義をしている.
第3章では既存の最大クリーク問題におけるアルゴリズムの導入とそれを
応用した最大2-plex問題と最大$k$-plex問題のアルゴリズムの提案を行っている.
第4章ではまとめと今後の課題について述べている.

\newpage

\chapter{諸定義}

\section{グラフの構造}
本論文中のグラフ$G=(V,E)$は頂点数$n=|V|$と辺の本数$m=|E|$を持つ単純無向
グラフとする.グラフ$G$の頂点部分集合$S$によって誘導される誘導部分グラフを
$G[S]$と表す.頂点$v$に辺が接続されているとき,$v$に隣接しているといい,
$v$に隣接している頂点を$v$の近傍と呼び,$N(v)$と表す.
頂点$(u,v)$間の最短パスの本数を頂点$(u,v)$の距離とする.

\section{最大クリークと最大$k$-plex}
グラフ$G$の頂点部分集合$S$が完全グラフとき,$S$をクリークといい,
クリークのうち最大サイズのものを見つける問題を最大クリーク問題という.
一方,$G[S]$の全ての頂点が少なくとも$|S| - k$の次数を持つとき,
$S$を$k$-plexといい,$k$-plexのうち最大サイズのものを見つける問題を
最大$k$-plex問題という.

%\section{準指数時間アルゴリズム}
%$o(n)$アルゴリズム

\newpage

\chapter{最大2-plex問題と最大$k$-plex問題の \\ 準指数時間アルゴリズム}

\section{最大クリーク問題}

以下に,既存の最大クリーク問題に関する定理とその証明を示す. \cite{fomin2010exact}
\begin{theorem} \label{theorem:1}
$m$本の辺を持ったグラフの最大クリーク問題は$2^{O(\sqrt{m})}$時間で解くことができる.
\begin{prf*}
$G$を連結グラフとする.最小次数の頂点$v$を選ぶ.もし$v$の次数が$\sqrt{m}$以上であるならば,
\[ 2m = \sum_{v \in V}^{} |N(v)| \geqq n\sqrt{m} \]
であるから$n \leqq 2\sqrt{m}$である.全ての可能性のある頂点集合を計算する
ブルートフォースアルゴリズムの実行時間は$n2^{O(n)} = n2^{O(\sqrt{m})}$となる.
以下$v$の次数は$\sqrt{m}$未満であるとする.

$G$中の最大サイズのクリーク$C$を見つけるブランチングアルゴリズムを考える.
以下のような1つの部分問題にブランチする.
\begin{enumerate}
 \item $v$が$C$中の頂点の一つである
 \item $v$が$C$中の頂点の一つでない
\end{enumerate}
1つ目の部分問題では$|N(v)|$の全ての頂点部分集合をブルートフォースによって
探索して$v$を含んだ最大サイズのクリークを見つける.2つ目の部分問題では$v$を
$G$から削除してアルゴリズムを再帰的に呼び出す.

このアルゴリズムのステップ数が多くとも$n2^{\sqrt{m}}$になることを$m$への帰納法によって証明する.
\begin{itemize}
 \item $m = 0$のとき,$G$はただ1つのクリークを持ち,その大きさは1である.	
 \item $v \in C$の場合に対応した部分問題を解くために,ブルートフォースを使って
$|N(v)|$中の最大サイズのクリーク$C'$を選ぶ.$C = C'  \cup  v$である.$|N(v)| < \sqrt{m}$で
あるから,必要なステップ数は多くとも$2^{\sqrt{m}}$ステップである.
 \item $v \notin C$の場合,帰納法の仮定によって問題は多くとも$(n - 1)2^{\sqrt{m}}$ステップで解ける.
\end{itemize}
ゆえにアルゴリズムの合計のステップ数は
\[  2^{\sqrt{m}} +  (n - 1)2^{\sqrt{m}} = n2^{\sqrt{m}} \]
でありその実行時間は $2^{O(\sqrt{m})}$である. 
\end{prf*}
\end{theorem}

\section{最大2-plex問題} \label{section:2plex}
最大2-plex問題の準指数時間アルゴリズムを提案するために,以下に2つの補題を示す.
\begin{lemma} \label{lemma:1}
$S$を2-plexの頂点集合とする.ある頂点$v$が$S$に含まれているとき,
$v$から距離が3以上離れた頂点は$S$に含まれない.
\begin{prf*}
$v$は2-plexの頂点集合$S$に含まれているとする.
$v$から距離が3離れた頂点を$a$,$v$から距離が2離れており$a$と隣接している
頂点を$b$とする.2-plexの定義より,$S$中に含まれる全ての頂点は少なくとも
$|S| - 2$の次数を持つ,つまり$S$中に含まれる頂点は$S$中の$|S| - 2$頂点と
隣接しているはずである.もし$a$が$S$中に含まれる頂点であると仮定すると,
$v$は$v, a, b$と隣接していないため矛盾する.したがって$a$は$S$に
含まれない.距離が4以上離れている頂点についても同様である.
\end{prf*}
\end{lemma}
\begin{lemma} \label{lemma:2}
$S$を2-plexの頂点集合とする.ある頂点$v$が$S$に含まれているとき,
$v$から距離が2離れた頂点は2つ以上$S$に含まれない.
\begin{prf*}
$v$は2-plexの頂点集合$S$に含まれているとする.
$v$から距離が2離れた2つの頂点をそれぞれ$a$,$b$とする.$S$中に含まれる頂点は
$S$中の$|S| - 2$頂点と隣接しているはずである.もし$a$と$b$が両方とも$S$中に
含まれると仮定すると,$v$は$v, a, b$と隣接していないため矛盾する.したがって
$a$と$b$は両方とも$S$に含まれることはない.
\end{prf*}
\end{lemma}
これらの補題をふまえ,最大2-plex問題に関する定理とその証明を示す.
\begin{theorem} \label{theorem:2}
$m$本の辺を持ったグラフの最大2-plex問題は$2^{O(\sqrt{m})}$時間で解くことができる.
\begin{prf*}
$G$を連結グラフとする.最小次数の頂点$v$を選ぶ.定理 \ref{theorem:1} と同様の議論により
$v$の次数は$\sqrt{m}$未満であるとする.

$G$中の最大サイズの2-plexである$S$を見つけるために2つの部分問題にブランチする.
\begin{enumerate}
 \item $v$が$S$中の頂点の1つである
 \item $v$が$S$中の頂点の1つでない
\end{enumerate}
1つ目の部分問題では$v$から距離2以内にあり,$S$に含まれる可能性のある
頂点部分集合を探索して$v$を含んだ最大2-plexを見つける.
2つ目の部分問題では$v$を$G$から削除してアルゴリズムを再帰的に呼び出す.

このアルゴリズムのステップ数が多くとも$n^{2}2^{\sqrt{m}}$になることを$m$への帰納法によって証明する.
\begin{itemize}
 \item $m = 0$のとき,$G$はただ1つの2-plexを持ち,その大きさは1である.	
 \item $v \in C$の場合に対応した部分問題を解くために,$v$を含んだ最大サイズの2-plexである$S$を選ぶ.
補題  \ref{lemma:1} より,$v$から距離3以上離れている頂点は$S$に含まれない.
補題  \ref{lemma:2} より,$v$から距離2離れている頂点は2つ以上は$S$に含まれない.
$v$の近傍の頂点数を$M$,$v$から距離2離れた頂点数を$N$とすると,$S$を選ぶのに必要な頂点数は %後で追記するかも
$2^{M} \times N$である.$M < \sqrt{m}$,$N \leqq n$であるから,必要なステップ数は
多くとも$2^{\sqrt{m}}$ステップである.
 \item $v \notin C$の場合,帰納法の仮定によって問題は多くとも$(n - 1)n2^{\sqrt{m}}$ステップで解ける.
\end{itemize}
ゆえにアルゴリズムの合計のステップ数は
\[  n2^{\sqrt{m}} +  (n - 1)n2^{\sqrt{m}} = n^{2}2^{\sqrt{m}} \]
でありその実行時間は $2^{O(\sqrt{m})}$である.
\end{prf*}
\end{theorem}

\section{最大$k$-plex問題} \label{section:kplex}
 \ref{section:2plex} では最大2-plex問題のついての準指数時間アルゴリズムの存在について述べた.
これを応用して一般の最大$k$-plex問題での準指数時間アルゴリズムの存在を示す.

まず, 補題  \ref{lemma:2} を一般の場合に拡張する.

\begin{lemma} \label{lemma:3}
$S$を$k$-plexの頂点集合とする.ある頂点$v$が$S$に含まれているとき,
$v$から距離が2以上離れた頂点は$k$つ以上$S$に含まれない.
\begin{prf*}
$v$は2-plexの頂点集合$S$に含まれているとする.$v$から距離2以上離れた$k$頂点
$a_{1}, a_{2}, \dots, a_{k}$が全て$S$に含まれていると仮定する.$k$-plexの定義より,
$S$中に含まれる全ての頂点は少なくとも$|S| - k$の次数を持つ,つまり$S$中に含まれる
頂点は$S$中の$|S| - k$頂点と隣接しているはずである.
しかし$v$は$v, a_{1}, a_{2}, \dots, a_{k}$の$k + 1$頂点と隣接していないので,矛盾する.
したがって$a_{1}, a_{2}, \dots, a_{k}$は全て$S$に含まれることはない.
\end{prf*}
\end{lemma}

%ここらへん曖昧
これらの補題をふまえ,最大$k$-plex問題に関する定理とその証明を示す.
\begin{theorem} \label{theorem:3}
$m$本の辺を持ったグラフの最大$k$-plex問題を
$O(n^{k}2^{\sqrt{m}})$時間で解くアルゴリズムが存在する.
\begin{prf*}
$G$を連結グラフとする.最小次数の頂点$v$を選ぶ.定理 \ref{theorem:1} と同様の議論により
$v$の次数は$\sqrt{m}$未満であるとする.

$G$中の最大サイズの$k$-plexである$S$を見つけるために2つの部分問題にブランチする.
\begin{enumerate}
 \item $v$が$S$中の頂点の1つである
 \item $v$が$S$中の頂点の1つでない
\end{enumerate}
1つ目の部分問題では$S$に含まれる可能性のある頂点部分集合を探索して
$v$を含んだ最大$k$-plexを見つける.
2つ目の部分問題では$v$を$G$から削除してアルゴリズムを再帰的に呼び出す.

このアルゴリズムのステップ数が多くとも$n^{k}2^{\sqrt{m}}$になることを$m$への帰納法によって証明する.
\begin{itemize}
 \item $m = 0$のとき,$G$はただ1つの$k$-plexを持ち,その大きさは1である.	
 \item $v \in C$の場合に対応した部分問題を解くために,$v$を含んだ最大サイズの$k$-plexである$S$を選ぶ.
補題  \ref{lemma:3} より,$v$から距離2以上離れている頂点は$k$つ以上は$S$に含まれない.
$v$の近傍の頂点数を$M$,$v$から距離2以上離れた頂点数を$N$とすると,$S$を選ぶのに必要な頂点数は
\[ 2^{M} \times  \sum_{i = 1}^{k - 1}\binom{N}{i}  \]
である.$M < \sqrt{m}$,$N \leqq n$ であり,また
\[ \sum_{i = 1}^{k}\binom{N}{i}  \leqq {\left( \frac{en}{k} \right)}^{k} \]
であることが知られているので必要なステップ数は多くとも${\left( \frac{en}{k} \right)}^{k}2^{\sqrt{m}}$ステップである.
 \item $v \notin C$の場合,帰納法の仮定によって問題は
多くとも$(n - 1){\left( \frac{en}{k} \right)}^{k}2^{\sqrt{m}}$ステップで解ける.
\end{itemize}
ゆえにアルゴリズムの合計のステップ数は
\[  {\left( \frac{en}{k} \right)}^{k}2^{\sqrt{m}} +  (n - 1){\left( \frac{en}{k} \right)}^{k}2^{\sqrt{m}} =n{\left( \frac{en}{k} \right)}^{k}2^{\sqrt{m}} \]
でありその実行時間は $O(n{\left( \frac{en}{k} \right)}^{k}2^{\sqrt{m}}) = O(n^{k}2^{\sqrt{m}})$である.
\end{prf*}
\end{theorem}

\newpage

\chapter{まとめと今後の課題}
\section{まとめ}
今回の研究では,既存の$2^{O(\sqrt{m})}$時間の最大クリーク問題のアルゴリズムを
応用して最大$k$-plexを解くアルゴリズムを考えた.その結果,最大$k$-plex問題を
$O(n^{k}2^{\sqrt{m}})$時間で解くアルゴリズムを得ることができた.

\section{今後の課題}
補題  \ref{lemma:2} を応用すると,$v$から距離$k$より離れた点は最大$k$-plexの集合に
含まれないことが示せる. \ref{section:2plex} のアルゴリズムでは,最大$k$-plexの集合に
含まれることがない頂点も探索しているので,無駄が存在する.この無駄を解消するために,
頂点を$v$からの距離によって場合分けして探索範囲を減らすアルゴリズムを考えたが,
計算時間$O(n^{k}2^{\sqrt{m}})$の境界を破ることはできなかった.
$v$からの距離以外の別のアプローチによって$O(n^{k}2^{\sqrt{m}})$の境界を突破するとこが
今後の課題である.

\newpage

\chapter*{謝辞}
本研究の機会を与え,数々の御指導を賜りました泉泰介准教授に深く感謝致します.
また,本研究を進めるにあたり多くの助言を頂き,様々な御協力を頂きました泉研究室
の学生のみなさんに深く感謝致します.

\newpage

\bibliography{b4sato}

\end{document}