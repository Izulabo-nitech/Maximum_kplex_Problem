%%
%% H27年度情報工学科卒業研究概要スタイルファイル
%%
\documentclass[a4j,twoside]{jarticle}
\bibliographystyle{unsrt}

\input{jdummy.def} % フォント関連のエラー対策(らしい)

%\usepackage{graphicx}
\usepackage[dvipdfmx]{graphicx}
\usepackage{amsmath,amsthm}		% math系
\usepackage{amssymb}			% math系
%\usepackage{float}				% 図表の挿入箇所を固定する[H]指定
%\usepackage{cite}				% 参考文献
%\usepackage{url}				% 参考文献中のURL表記
\usepackage{algorithm}			% アルゴリズム環境
\usepackage{algorithmic}		% アルゴリズム環境
\usepackage{comment}			% コメントアウト環境
\usepackage{bm}	%太字形式のベクトル

% 定理環境
\newtheorem{theorem}{定理}
\newtheorem{lemma}{補題}
\newtheorem{definition}{定義}


% Unix上でのコンパイルはthesis_abst-jisを利用してください.
\usepackage{thesis_abst}

% マージンはプリンタによって変更
\addtolength{\oddsidemargin}{0mm}
\addtolength{\evensidemargin}{0mm}

% baselinestretchを変更すると上部枠の大きさが変わるのでおすすめしない
\renewcommand{\baselinestretch}{1}

\種別{情 報 工 学 科 卒 業}  % この行を消してはいけない
\学籍番号{27115067}
\氏名{佐藤 僚祐}
%\英語氏名{} %未使用
\研究室{泉}
\系{ネットワーク} % 学生が所属する系を記入.教員(研究室)の所属する系ではない
\題目{密でないグラフに対して最大k-plexを発見する \\ 準指数時間アルゴリズム} % 途中で改行 "\\" を挿入可
\年度{30} % !=年 発表は2月です
\begin{document}              % この行を消してはいけない
\twocolumn[\vspace*{9mm}]     % この行を消してはいけない
\begin{論文概要}              % この行を消してはいけない
\setcounter{page}{2}          % 表(左綴じ)は1、裏(右綴じ)は2を指定
%%%%%%% ここからアブスト本体 %%%%%%%

\section{はじめに}
ソーシャルネットワーク解析等に代表されるように,グラフデータのマイニング分野において,
辺密度の高い部分グラフを見つけることは大きな研究のトピックと
なっている\cite{washio2003state}.辺密度の高い部分グラフとして,最も一般的に
使用されているモデルはクリークである.クリークとは全ての2頂点間に辺がある部分グラフで,
辺密度最高の部分グラフとみることができる.グラフ中の最も大きいサイズのクリークを
見つける最大クリーク問題はグラフアルゴリズムにおける基本的な問題であるが,
一般に厳密解,近似解を問わず,その発見問題はNP完全である.また,クリーク発見における
問題として,ほとんど同じ頂点集合からなるクリークを多数発見してしまうことが挙げられる.
そこで,クリークを緩和した,高密度部分グラフのさまざまな定義が提案されている.
$k$-plexはそのような定義の一つである.サイズ$n$の$k$-plexとは,すべての
頂点の次数が$n-k$であるようなグラフである.$k=1$のとき1-plexはクリークと同
義である.グラフ中の最も大きいサイズの$k$-plexを
見つける問題を最大$k$-plex問題という.最大
$k$-plex問題は最大クリーク問題の一般化であるため自明にNP完全であり,
頂点数$n$に対して$n$の多項式時間で解くことは絶望視されている.
そのため,最大$k$-plex問題を高速に解くための指数時間厳密アルゴリズムの
研究が行われており,これまでにさまざまなアルゴリズムが提案されている.
提案されたアルゴリズムの多くは$2^{(1-\epsilon)n}n^{O(1)}$時間で動作する.
ここで$\epsilon$は$k$の値によって決まる微小定数である.

本研究では,グラフの頂点数でなく,辺の本数をパラメタとした計算時間を持つ
厳密アルゴリズムの検討を行う.最大クリーク問題については,辺の本数$m$のグラフに対して
計算時間が$O(2^{O(\sqrt{m})})$時間となるようなアルゴリズムが知られている \cite{fomin2010exact} 
が本研究では最大$k$-plex問題に対して同様のアルゴリズムが設計可能であることを示す.
具体的には,辺の本数$m$に対して最大$k$-plex問題を$O(n^{k}2^{\sqrt{m}})$時間で
解くアルゴリズムを提案する.このアルゴリズムは$m$の準指数時間で動くため,
辺の本数が少ないグラフに対する最大$k$-plexの発見に有用である.  

\section{諸定義}
グラフの構造を次のように定義する.グラフ$G=(V,E)$は
頂点数$n=|V|$と辺の本数$m=|E|$を持つ単純無向
グラフとする.グラフ$G$の頂点部分集合$S$によって
誘導される誘導部分グラフを$G[S]$と表す.頂点$v$に
辺が接続されているとき,$v$に隣接しているといい,
$v$に隣接している頂点を$v$の近傍と呼び,$N(v)$と
表す.頂点$(u,v)$間の最短パスの本数を頂点$(u,v)$の
距離とする.

\section{最大$k$-plex問題の \\ 準指数時間アルゴリズム}
我々が提案する最大$k$-plexの発見アルゴリズムは,
最大クリーク問題を$O(2^{O(\sqrt{m})})$で解くことが
できるという既存の結果を応用している. 
以下に本研究のアルゴリズムとその実行時間を示す.

最小次数の頂点$v$を選ぶ($v<\sqrt{m}$が仮定できる).
$G$中の最大サイズの$k$-plexである$S$を見つける
ために2つの部分問題にブランチする.
\begin{enumerate}
 \item $v$が$S$中の頂点の1つである
 \item $v$が$S$中の頂点の1つでない
\end{enumerate}

1つ目の部分問題では$S$に含まれる可能性のある
頂点部分集合を探索して$v$を含んだ
最大$k$-plexを見つける.補題により距離2以上離れている
頂点は$k$つ以上は$S$に含まれないことが示せるので,
$S$を選ぶのに必要な頂点数は
\[ 2^{M} \times  \sum_{i = 1}^{k - 1}\binom{N}{i}  \]
である($M$:$v$の近傍の頂点数,$N$:$v$から距離2以上
離れた頂点数).$M < \sqrt{m}$,$N \leqq n$ であり,また 
$ \sum_{i = 1}^{k}\binom{n}{i} \leqq ({en}/{k})^{k}$
であることが知られているので必要なステップ数は多くとも
$({en}/{k})^{k}2^{\sqrt{m}}$ステップである.

2つ目の部分問題では$v$を$G$から削除してアルゴリズムを
再帰的に呼び出す.再帰の回数は多くとも$n-1$回である.

ゆえにアルゴリズムのステップ数は
\[  {\left( \frac{en}{k} \right)}^{k}2^{\sqrt{m}} +  (n - 1){\left( \frac{en}{k} \right)}^{k}2^{\sqrt{m}} =n{\left( \frac{en}{k} \right)}^{k}2^{\sqrt{m}} \]
でありその実行時間は 
\[ O(n{\left( \frac{en}{k} \right)}^{k}2^{\sqrt{m}}) = O(n^{k}2^{\sqrt{m}}) \]
である.

\begin{comment}
\section{まとめと今後の課題}
今回の研究によって,最大$k$-plex問題を$O(n^{k}2^{\sqrt{m}})$時間で
解くアルゴリズムを得ることができた.このアルゴリズムでは
$S$に入る可能性のない頂点も探索しているので,これを改良して
$O(n^{k}2^{\sqrt{m}})$の境界を突破できるかが今後の課題となる.
\end{comment}

\bibliography{b4sato}

%%%%%% 以下の行は消さないこと %%%%%%%

\clearpage                       % この行を消すと最終ページの枠線消滅の危機
\end{論文概要}                   % この行を消してはいけない
\end{document}                   % この行を消してはいけない
